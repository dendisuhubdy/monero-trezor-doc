\documentclass[]{article}
\usepackage{cite}
\usepackage{url}

% TODO packages
% https://tex.stackexchange.com/questions/9796/how-to-add-todo-notes
\usepackage{xargs}                      % Use more than one optional parameter in a new commands
\usepackage[pdftex,dvipsnames]{xcolor}  % Coloured text etc.
\usepackage[colorinlistoftodos,prependcaption,textsize=tiny]{todonotes}
\newcommandx{\unsure}[2][1=]{\todo[linecolor=red,backgroundcolor=red!25,bordercolor=red,#1]{#2}}
\newcommandx{\change}[2][1=]{\todo[linecolor=blue,backgroundcolor=blue!25,bordercolor=blue,#1]{#2}}
\newcommandx{\info}[2][1=]{\todo[linecolor=OliveGreen,backgroundcolor=OliveGreen!25,bordercolor=OliveGreen,#1]{#2}}
\newcommandx{\improvement}[2][1=]{\todo[linecolor=Plum,backgroundcolor=Plum!25,bordercolor=Plum,#1]{#2}}
\newcommandx{\thiswillnotshow}[2][1=]{\todo[disable,#1]{#2}}

%opening
\title{Monero + Trezor}
\author{Du\v{s}an Klinec}

\begin{document}

\maketitle

\begin{abstract}
Design of the Monero integration to the Trezor environment and reuired changes in the Monero codebase.
\end{abstract}

\section{Introduction}

\subsection{Notation}

Basic system:

\begin{itemize}
	\item $G$ is a base point of the curve ed25519 $E$, of order $l$
	\item $H_p : \{0,1\}^* \rightarrow E$, hash function to curve point
	\item $H_s : \{0,1\}^* \rightarrow [1, l-1]$, hash function to the scalar
	\item $H = H_p(G)$, a point on the curve $E$ where $H=hG$, $h$ is unknown
	\item $(a, A)$, $(b, B)$ account view-key / spend-key pair
\end{itemize}
\\

Transactions:

\begin{itemize}
	\item $(r, R)$ transaction key 
	\item $(r, R)$ transaction key pair
\end{itemize}


%\improvement[inline]{Add more}

% References
\bibliography{monero}{}
\bibliographystyle{plain}

\end{document}

